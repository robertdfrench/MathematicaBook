\chapter*{If you have never used \emph{Mathematica} \ldots }
\label{chap:Prelim}

This pamphlet assumes you have done at least a little programming before, possibly in C, C++, FORTRAN, Visual Basic, or some other language. If not, \emph{that is completely fine}. We will walk you through a little basic programming here so that you can get your feet wet, and then you will be ready to take on chapter 1!

\section{Notebooks and Cells}

Notebooks contain many cells. Cells contain code.

\section{Fancy Typing and the Palette}

You can make your mathematica code look more mathematical with this stuff

\subsection{Subscripts, Exponents, and Fractions}

For the sake of clarity, or to enhance the presentation of your code, you may want to make it look a bit more like normal mathematics. For example, to write $x^2$ in Mathematica, type ``x'' and then ``Ctrl+6'' and it will elevate the cursor and create a box above the x.

\subsection{$\pi$, $\theta$, and $alpha$}

Using the ``Esc'' button, you can generate these symbols. ``Esc+pi+Esc'' will get you the $\pi$ symbol.

\subsection{Sums, Integrals, and Derivatives}

\section{Functions}
\emph{Mathematica} has tons of built-in functions. It has trig functions like \texttt{Sin} and \texttt{Cos}. It also has functions for plotting graphs and generating lists or tables of data (see ``\nameref{chap:Plotting}'' and ``\nameref{chap:ListsMap}'').
