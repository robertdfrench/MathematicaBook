\chapter{Finding Help in \emph{Mathematica} }
\label{chap:Help}

\section{Documentation Center}

You can access \emph{Mathematica}'s built-in documentation center by pressing ``Shift'' + ``F1'' from anywhere in the application. The Help Center provides the official documentation for all fo the built-in functions you will use, and for many additional packages that are included in \emph{Mathematica} but not enabled by default.

Each function has its own page in the Help Center. Try searching for \texttt{Plot}. You will see several things: 
\begin{enumerate}
	   \item A summaray of usage detailing all the ways that \texttt{Plot} can be invoked, and what the differences are.
	   \item A section showing further examples and more sophisticated tricks.
	   \item A list of related functions, such as \texttt{Plot3D} and \texttt{Manipulate}.
\end{enumerate}
One of the difficulties in reading the documentation given in the Help Center is that it references other documentation in the Help Center, and you may get the feeling that you are going in circles. This is quite normal, and no cause for worry. Every programmer encounters this difficulty when learning a new language, though they may not admit it. The trick really is to keep following the trail until you build up enough context that what you are reading starts to make sense.

It is important, in situations like these, to put aside (momentarily) the problem for which you are searching and focus on what is presented in the documentation. Play with the examples that are given. Try changing parts of them, rearranging the arguments, throwing in different numbers or values. If you mess it up, you can always just close the Help Center and come back to it later. Above all, it is vital to \emph{be patient}. 

The single most frustrating experience you can have while programming is to spend ages hunting through the documentation for \emph{The Wrong Way to Do It}. The docs, you see, are designed to show \emph{The Right Way to Do It}, and it may sometimes take a while to recognize that for what it is.

You may trust that the authors have spent many long sessions dumbfounded in front of the \emph{Mathematica} Help Center only to realize in a great flurry of excitement that the answer is completely different than we initially expected. One cannot rush that sort of thing; this stuff just takes time.

\section{Stack Overflow}
One of the greatest resources for \emph{Mathematica} help on the internet is \href{http://stackoverflow.com/questions/tagged/mathematica}{StackOverflow/Mathematica}. Also becoming very popular is \href{http://mathematica.stackexchange.com/}{mathematica.stackexchange.com}, which is a separate site dedicated enitrely to \emph{Mathematica} issues. Both of these should serve you well in your quest to find answers to your problems.

StackOverflow is a forum in which users are encouraged to give insightful answers in order to receive \emph{points}, which amount to social capital. Thus, the answers you will find on StackOverflow are consistently of a higher quality than those you will find on other programming-related websites.

On StackOverflow, all questions are \emph{tagged} according to which programming language or platform they pertain to. The link above will take you directly to the \emph{Mathematica} questions, and from there you can search for more specific information about your question.

Generally, the probability that the issue you have run into when programming in \emph{Mathematica} (or any language) is unique is very low, so the odds are in your favor that someone has encountered a very similar problem before you. Thus, the challenge is to alter your search terms judiciously until you stumble on a problem that seems to fit the issue you are dealing with.

One way to help with this process is to speculate a few guess about what the problem might be. For example, are you using some functions whose behavior you don't quite understand? Maybe you getting a weird output that doesn't look like what you think it should? Searching on StackOverflow for things like ``ListPlot no graph'' is more likely to get you useful results than ``no graph''. 

\subsection{Asking Questions on Stack Overflow}
Accounts on StackOverflow are free, and joining this community will help you learn a great deal about both \emph{Mathematica} and programming in general. Searching through other people's questions and the suggested answers can be very informative, and it is a good intellectual exercise to try to solve some on your own (and you may even be rewarded with profile points!).

Of course, membership on StackOverflow also allows you to post questions. Understand though that the community expects you to have done some work before you post a question. Generally, before posting new questions, it is advisable to:

\begin{enumerate}
	   \item Try your query on a major search engine like Google or Bing. At least try all the links on the first page to see if they have anything helpful to offer.
	   \item Try searching WolframAlhpa. While this is not exactly a search engine in the usual sense, its results usually contain \emph{Mathematica} code which is occasionally helpful.
	   \item Try the \emph{Mathematica} documentation. You can access this by pressing Shift+F1 while running \emph{Mathematica}. Specifically, you will want to look at the examples that are available on the documentation page for each function that you might have questions about. Frequently functions can take different arguments, or give different output depending on parameters or options, so it may be that you need to invoke your function in a slightly different way. 
\end{enumerate}

If you have done these things and still not found the answer to your question, then it will be cool to post your question on Stack Overflow. One thing to note is that while people on Stack Overflow are usually very eager to help, they need enough information about what you are trying to do to be able to understand where you might be running into trouble. Generally, posting a single line of code may not be enough. Also, keep in mind that folks on the internet are very unlikely to be familiar with your research, so it is important to track down the issue as specifically as possible.

