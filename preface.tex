\chapter*{Preface}

This book is written especially for the students of MATH 3120 and 3130 at Austin Peay State University in Clarksville, TN. It is intended to introduce the reader, whether versed in programming or not, to the basic elements of \emph{Mathematica}.

Computer algebra systems and symbolic programming provide a valuable means of exploring research topics at the undergraduate level. In addition, \emph{Mathematica}'s facilities for functional programming provide a means for students to learn to solid programming habits.

This book features an introduction to basic usage of \emph{Mathematica} for those who have never used it before, and for those who have never done any programming before. It is our aim that this preliminary section ``\nameref{chap:Prelim}'' will provide enough background to get started with the remainder of the book.

We hope that this book will serve to clarify and make \emph{Mathematica} a more accessible programming language. We hope this book illuminates the joy of programming, and that those who read this book will go on to publish research of their own.\\
\\
Samuel N. Jator \& Robert D. French\\
Austin Peay State University\\
\today
