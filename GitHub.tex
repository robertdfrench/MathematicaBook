\chapter{Using GitHub to Collaborate with Teammates}

If you are working on a research project with a team, you may decide simply to email the mathematica notebooks you are using back and forth every time either of you makes a change. This seems like a good idea, and indeed it is the simplest way to share changes, but it can lead to some surprisingly hairy problems very quickly.

For example, let us suppose that Nell, Ben, and Raman are on a team together. They start with one document, \emph{research1.nb}, which contains all of their initial code that they plan to use for their project, and each person takes a copy of the document home with them. Later that week, Nell and Ben both make separate changes to the document and email their new documents to Raman. When Raman checks her email, she sees that Nell and Ben have both come up with neat ideas, but in order to get them to work together, she has to cut and paste their changes by hand. This seems straightforward enough, until Raman tries to run the code and she encounters a bug!

What might have happened? Did she make a typo while rearranging the code? Do Nell and Ben's changes conflict in some subtle way? Did the documents they emailed her match their latest changes (that is, did either Nell or Ben accidentally send an old document)? Emailing documents leads to many, many difficulties of this nature. Fortunately, programmers before us have encountered this same issue, and developed tools to solve it.

\section{The Value of Version Control}

Version Control is, in general, a way of keeping track of changes to files across a team of people (even if it is only a team of one!). Version Control systems are programs that you can install on your computer that will note the changes you make to your code in a way that allows you to access specific versions of your code at any time.

At first, this may sound like a simple backup system, and version control can certainly act as a backup system, but there are more powerful features than that. For example, version control systems allow you to maintain \emph{multiple histories} of a file. This can be useful when you want to experiment with your code. Nell may be brainstorming an improvement she wants to make to the way that graphs are stored, using a SparseMatrix instead of a List of Lists. With version control, she can ``branch'' from the current development history into a separate \emph{parallel} history so that her experiment will not affect Ben and Raman. If Nell's idea works, she can ``merge'' her branch back into the main history. If it doesn't, she can simply discard it. Either way, it is clear that trying this with a backup system or a set of shared folders would lead to a disorganization nightmare for the team.

\section{Using Git}

Git is an excellent version control system. You can learn how to install it by going to \href{http://git-scm.org}{git-scm.org}. The basic operations in Git are as follows:

\begin{code}
	   # Creates a new repository
	   \$ git init
\end{code}
Which will create a new repository. The repository is simply the database that keeps track of changes you make to all the files in a given folder. When you make a change to a file, tell Git about it by typing
\begin{code}
	   \$ git add BenNellRaman_Research.nb
\end{code}
when you have made a set of changes that you would like to be stored in the repository's history, you need to ``commit'' those changes:
\begin{code}
	   \$ git commit -am ``Storage Algorithm now uses SparseMatrix for more efficiency''
\end{code}

\section{Sharing Repositories on GitHub} 

The easiest way to keep a shared repository for your team is for everyone to create an account on \href{http://www.github.com}{GitHub}. GitHub is a free service for hosting open source projects using git. 

TODO setting up repo

TODO giving push access
