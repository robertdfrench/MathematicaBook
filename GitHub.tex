\chapter{Using GitHub to Collaborate with Teammates}

If you are working on a research project with a team, you may decide simply to email the mathematica notebooks you are using back and forth every time either of you makes a change. This seems like a good idea, and indeed it is the simplest way to share changes, but it can lead to some surprisingly hairy problems very quickly.

For example, let us suppose that Nell, Ben, and Raman are on a team together. They start with one document, \emph{research1.nb}, which contains all of their initial code that they plan to use for their project, and each person takes a copy of the document home with them. Later that week, Nell and Ben both make separate changes to the document and email their new documents to Raman. When Raman checks her email, she sees that Nell and Ben have both come up with neat ideas, but in order to get them to work together, she has to cut and paste their changes by hand. This seems straightforward enough, until Raman tries to run the code and she encounters a bug!

What might have happened? Did she make a typo while rearranging the code? Do Nell and Ben's changes conflict in some subtle way? Did the documents they emailed her match their latest changes (that is, did either Nell or Ben accidentally send an old document)? Emailing documents leads to many, many difficulties of this nature. Fortunately, programmers before us have encountered this same issue, and developed tools to solve it.

\section{The Value of Version Control}

Version Control is, in general, a way of keeping track of changes to files across a team of people (even if it is only a team of one!). Version Control systems are programs that you can install on your computer that will note the changes you make to your code in a way that allows you to access specific versions of your code at any time.

\section{Using Git}

Git is awesome.

\section{Sharing Repositories on GitHub}

Use GitHub or git left behind.
