\chapter{Lists and Map}
\label{chap:ListsMap}
This chapter introduces what computer scientists refer to as \emph{functional programming}. To understand some of the impact of this, we begin by discussing the fundamental data structure of functional programming, the List.

A List is similar to an \emph{array} that you might have encountered in other programming languages. One of the main differences is that Lists are designed to grow, whereas arrays are designed to take up a fixed amount of memory.

Lists in \emph{Mathematica} can be constructed very simply by the following statement:
\begin{code}
	   A = List[];
\end{code}
or equivalently
\begin{code}
	   A = {};
\end{code}
however, the former style should be preferred as it is more explicit\footnote{Good programmers always aim for their code to be \emph{clear} and \emph{explicit}. This makes it easier for others to read their code and understand its meaning, and that is good for friendship}. Lists can be grown by \emph{appending} to elements to them. For example, in order to create the list $1,2,3,4$, we could do the following:
\begin{code}
	   AppendTo[A, 1];
	   AppendTo[A, 2];
	   AppendTo[A, 3];
	   AppendTo[A, 4];
\end{code}

Of course, we could also define this list explicitly as follows:
\begin{code}
	   A = {1,2,3,4};
\end{code}
and this is of course much more concise. Generally, if a list can be defined without doing any calculations, i.e. if it is a constant, you will define it all at once as we have done here. However, if the list must be built up programmatically, it is necessary to use the \texttt{AppendTo} function as described above.

We can also access list elements directly by using the \texttt{[[]]} operator. For example, to get the first element of \texttt{A}, we do as follows:
\begin{code}
	   A[[1]]
\end{code}
Note that unlike most programming languages, \emph{Mathematica} begins indexing lists with \texttt{1} rather than \texttt{0}. This is good, because it is in keeping with most mathematical notation for dealing with vectors and matrices. We can also assign values explicitly to a list by using the \texttt{[[]]=} operator like so:
\begin{code}
	   A[[2]] = 5;
\end{code}

TODO: Can we use DelaySet on a list element? What is the effect of this?

You cannot grow a list in this fashion. That is, for a list like \texttt{A} that has 4 elements, you cannot add a fifth element by 
\begin{code}
	   A[[5]] = 5.0;
\end{code}
This is because, like arrays in C or Java, lists have a fixed width, and \emph{Mathematica} will not allocate more memory for them unless you explicitly tell it to (using the \texttt{AppendTo} function).\footnote{This is a much different model than languages like Perl or Ruby that automatically grow an array when you make an ``out of bounds'' assignment. This strategy, while handy, can slow down your program because memory will be allocated frequently in small chunks rather than infrequently in large chunks.}
