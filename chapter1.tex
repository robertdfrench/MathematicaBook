\chapter[Stuff You've Seen Before]{A Review of Some Stuff You've Probably Seen Before}
\label{chap:SeenBefore}

Before beginning this chapter, it is assumed that you know a few things about programming already. Specifically, you will need to know how to
\begin{enumerate}
	   \item assign values to variables
	   \item evaluate \emph{Mathematica} cells
\end{enumerate}
and if you don't, that's quite alright. Just see the section ``\emph{\nameref{chap:Prelim}}'' on page \pageref{chap:Prelim}, and even if you have done a bit of programming before, we will walk you through some basic \emph{Mathematica}.

\section{Variable Assignment}
The first thing we need to know in any language is how to assign variables. Probably you are thinking ``\emph{I learned this in CSCI 1010!}'', but \emph{Mathematica} is a subtle language, and does not always work as you might expect if you are coming from C++ or FORTRAN.

\begin{verbatim}
i = 1;
j := 2;
k = 3
l := 4
\end{verbatim}

Looking at this example, we can see that there are four slightly different ways to ``assign'' values to a variable, so let's discuss this a bit. If you put this code into a \emph{Mathematica} cell, you will see that, upon evaluating the cell, 
