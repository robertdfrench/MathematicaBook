\chapter{Defining Your Own Functions}
\label{chap:Functions}

Defining one's own functions is very easy in \emph{Mathematica}. For example, suppose you want a function that converts from Cartesian coordinates to polar coordinates. This is a map of the form $(x,y) \mapsto (r, \theta)$ using the relationships $x = r\cos(\theta)$ and $y = r\sin(\theta)$. We also want an inverse function of the form $(r, \theta) \mapsto (x,y)$ that will convert Polar coordinates to Cartesian. Let's see how this would look in code:

\begin{code}
	   (* Cell 1 *)\\
	   CartesianToPolar[\{x\_,y\_\}]:= \{$\sqrt{x^2 + y^2}$, ArcSin$\left[\frac{y}{\sqrt{x^2 + y^2}}\right]$\};\\
	   PolarToCartestian[\{r\_,$\theta$\_\}]:= \{r Cos[$\theta$], r Sin[$\theta$]\};\\
\end{code}

We can invoke these functions by using the \expr{[]} operator, as discussed in the section ``\nameref{sec:InvokingFunctions}''. Let's see what happens when we do. 

\begin{code}
	   (* Cell 2 *)\\
	   CartesianToPolar[\{1,1\}]\\
	   $\hookrightarrow$ \{$\sqrt{2}, \frac{\pi}{4}$\}\\
\end{code}

Since these two functions are one another's inverse, we can compose them and recover our original input:

\begin{code}
	   (* Cell 3 *)\\
	   point = \{1,1\};\\
	   PolarToCartesian[CartesianToPolar[point]];\\
	   $\hookrightarrow$ \{1, 1\}\\
\end{code}


\section{Using \expr{Block} for Scope}

Notice how, in the above definition of \expr{CartesianToPolar}, we calculate $x^2 + y^2$ in two different places. This isn't the end of the world, but it's a little inelegant, so let's clean it up by calculating it first and using it later as a variable:

\begin{code}
	   (* Cell 4 *)\\
	   CartesianToPolar[\{x\_,y\_\}]:= (\\
	   	r = $\sqrt{x^2 + y^2}$;\\
	   	\{r, ArcSin$\left[\frac{y}{r}\right]$\}\\
	   );\\
\end{code}

That looks much nicer. Cleaning up the code in this way will make your code more readable, and that will make it easier to debug and share with others. However, there is a small side effect: the value stored in \expr{r} will ``leak'' out of your function. To see what this means, let's do an example:

\begin{code}
	   (* Cell 5 *)\\
	   CartesianToPolar[{1,1}];\\
	   r\\
	   $\hookrightarrow$ $\sqrt{2}$
\end{code}

Now, why should \expr{r} even exist outside of where it is defined in the \expr{CartesianToPolar} function? That's because \emph{Mathematica} makes all new variables \emph{global} by default. This means that unless you indicate otherwise, variables created anywhere in your program will be accessible from anywhere else. Global variables are not necessarily bad, but if you don't keep an eye on them they can be a total nightmare.

Fortunately, \emph{Mathematica} provides a construct known as a \emph{block} which allows you to set up some local variables and automatically erase them when you're done. You can make a block by calling the \expr{Block} function like this:

\begin{code}
	   (* Cell 6 *)\\
	   CartesianToPolar[\{x\_,y\_\}]:= Block[{z},\\
			 z = $\sqrt{x^2 + y^2}$;\\
			 \{z, ArcSin$\left[\frac{y}{z}\right]$\}\\
	   ];\\
\end{code}

Now if you repeat the experiment from Cell 5, you will see that no new value is stored in the variable \expr{z}. That is because the block automatically set \expr{z} up as a local variable and erased it when the block ended. Keeping variables like this contained in a block will help keep your code clean, because it will keep cells from adversely interacting with one another. 

\section{Accepting Functions as Arguments}

Frequently it is useful to take a second function as an argument for a function you have defined. There are no specific use cases for this, but if you are familiar with the technique, you will be able to recognize when it will be helpful.

As an example, let's say that you are preparing a report and you would like all of your graphs to be displayed uniformly. You could manually go throuhg your code and set all of your \expr{Plot} settings to be the same, or you could define a single abstract function to take certain arguments and plot all of your graphs in the same style.

Let's assume that in this report, we will be plotting a comparions of \emph{Exact Solutions} versus \emph{Approximate Solutions} for a set of differential equations. That means we want our custom plot function to take two arguments
\begin{enumerate}
	   \item The Exact Solution as a \emph{Mathematica} function
	   \item The Approximate Solution as a list of points to plot
\end{enumerate}
so we know that the \emph{signature} of our function will look like the following:
\begin{code}
	   AwesomePlot[ExactSoln\_, ApproxSoln\_, LowerBound\_, UpperBound\_]:=
\end{code}

This is because we will need to accept not only the Exact Solution and the Approximate Solution, but also the interval on which those solutions should be plotted. This is part of the normal information we would pass to \expr{Plot}\footnote{You may recall that normally \expr{Plot} is invoked by using an expression, not a function variable, as the first argument (i.e. \expr{Plot[Sin[2 x], \{x,0,1\}]}). This is because \expr{Plot} evaluates its arguments in a non-standard way. TODO: List settings. It is defined in such a way that the first argument is ``Held''. TODO: Section on hold evaluations}. Now we want to display both of these functions on the same graph, preferebly in different colors, so that we may compare them. Thus the body of our function may look like:

\begin{code}
	   ExactGraph = Plot[ExactSoln[x], \{x, LowerBound, UpperBound\}, PlotStyle -> Blue];\\
	   \\
	   n = Length[ApproxSoln];\\
	   StepSize = (UpperBound - LowerBound)/(n - 1);\\
	   ApproxPoints = Table[\{ApproxSoln[[i]],(i - 1) * StepSize + LowerBound\},\{i,1,n\}];\\
	   ApproxGraph = ListPlot[ApproxPoints, PlotStyle -> Red];\\
	   Show[ExactGraph, ApproxGraph]
\end{code}

\MPlot{Functions}{AwesomePlotSin}{\expr{AwesomePlot[Sin,\{0,1,0,-1,0\},0,2$\pi$]}}
And we can see that this will form a very handy tool that will make sure all of the plots in your report are very consistent, and it will make your code much, much cleaner.

\section{Recursive Functions}

It is sometimes handy to define a function that turns around and calls itself.

\subsection{Calculating the Fibonacci Sequence}

Now in Number Theory, this is pretty interesting. But in a section on programming with recursive functions, it's really a pretty boring example. Let's do something cool

\subsection{Building your own Derivative operator}


\textbf{Exercises}.
\begin{enumerate}
	   \item In Cell 1, why were the functions defined with their variables inside a list (\{\})? What would be different if the curly braces were left off? How would this affect composition in Cell 3?
	   \item Consider the code given for \expr{AwesomePlot} and explain why \expr{N - 1} yields a more accurate \expr{StepSize} than \expr{N} would.
	   \item Expand the code for \expr{AwesomePlot} so that it takes a string to use as the title for the plot
	   \item Expand the code for \expr{AwesomePlot} so that it automatically labels plots as ``Figure 1'', ``Figure 2'', etc without explicitly taking an argument. Hint: It should contain a variable that gets incremented every time you call \expr{AwesomePlot}. 
\end{enumerate}
